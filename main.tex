\let\negmedspace\undefined{}
\let\negthickspace\undefined{}

\documentclass{beamer}
\usepackage{amsthm}
 \usepackage{gensymb}
 \usepackage{polynom}
\usepackage{amssymb}
%
  \usepackage{stfloats}
\usepackage{bm} 
 \usepackage{longtable}
 \usepackage{enumitem}
 \usepackage{mathtools}
 \usepackage{tikz}
  %  \usepackage[breaklinks=true]{hyperref}
  \usepackage{listings}
\usepackage{color}                                            
\usepackage{array}                                            
\usepackage{longtable}                                        
\usepackage{calc}                                             
     \usepackage{multirow}                                         
     \usepackage{hhline}                                           
     \usepackage{ifthen}                                           
     \usepackage{lscape}     
\usetheme{CambridgeUS}
\DeclareMathOperator*{\Res}{Res}
\DeclareMathOperator*{\equals}{=}
\renewcommand\thesection{\arabic{section}}
\renewcommand\thesubsection{\thesection.\arabic{subsection}}
\renewcommand\thesubsubsection{\thesubsection.\arabic{subsubsection}}
% \renewcommand\thesectiondis{\arabic{section}}
% \renewcommand\thesubsectiondis{\thesectiondis.\arabic{subsection}}
% \renewcommand\thesubsubsectiondis{\thesubsectiondis.\arabic{subsubsection}}
\hyphenation{op-tical net-works semi-conduc-tor}
 \def\inputGnumericTable{}                                 %%
\lstset{ 
frame=single,
breaklines=true,
columns=fullflexible
}

% \newtheorem{theorem}{Theorem}[section]
% \newtheorem{problem}{Problem}
% \newtheorem{proposition}{Proposition}[section]
% \newtheorem{lemma}{Lemma}[section]
% \newtheorem{corollary}[theorem]{Corollary}
% \newtheorem{example}{Example}[section]
% \newtheorem{definition}[problem]{Definition}
\newcommand{\BEQA}{\begin{eqnarray}}
\newcommand{\EEQA}{\end{eqnarray}}
\newcommand{\define}{\stackrel{\triangle}{=}}
\newcommand*\circled[1]{\tikz[baseline= (char.base)]{
    \node[shape=circle,draw,inner sep=2pt] (char) {#1};}}
\bibliographystyle{IEEEtran}
\providecommand{\mbf}{\mathbf}
\providecommand{\pr}[1]{\ensuremath{\Pr\left(#1\right)}}
\providecommand{\qfunc}[1]{\ensuremath{Q\left(#1\right)}}
\providecommand{\sbrak}[1]{\ensuremath{{}\left[#1\right]}}
\providecommand{\lsbrak}[1]{\ensuremath{{}\left[#1\right.]}}
\providecommand{\rsbrak}[1]{\ensuremath{{}\left[#1\right.]}}
\providecommand{\brak}[1]{\ensuremath{\left(#1\right)}}
\providecommand{\lbrak}[1]{\ensuremath{\left(#1\right.)}
\providecommand{\rbrak}[1]{\ensuremath{\left[#1\right.]}}}
\providecommand{\cbrak}[1]{\ensuremath{\left\{#1\right\}}}
\providecommand{\lcbrak}[1]{\ensuremath{\left\{#1\right.}}
\providecommand{\rcbrak}[1]{\ensuremath{\left.#1\right\}}}
\theoremstyle{remark}
\newtheorem{rem}{Remark}
\newcommand{\sgn}{\mathop{\mathrm{sgn}}}
\providecommand{\abs}[1]{\left\vert#1\right\vert}
\providecommand{\res}[1]{\Res\displaylimits_{#1}} 
\providecommand{\norm}[1]{\left\lVert#1\right\rVert}
\providecommand{\mtx}[1]{\mathbf{#1}}
\providecommand{\mean}[1]{E\left[ #1 \right]}
\providecommand{\fourier}{\overset{\mathcal{F}}{ \rightleftharpoons}}
\providecommand{\system}{\overset{\mathcal{H}}{ \longleftrightarrow}}
% \newcommand{\solution}{\noindent \textbf{Solution: }}
\newcommand{\cosec}{\,\text{cosec}\,}
\newcommand*{\permcomb}[4][0mu]{{{}^{#3}\mkern#1#2_{#4}}}
\newcommand*{\perm}[1][-3mu]{\permcomb[#1]{P}}
\newcommand*{\comb}[1][-1mu]{\permcomb[#1]{C}}
\renewcommand{\thetable}{\arabic{table}} 
\providecommand{\dec}[2]{\ensuremath{\overset{#1}{\underset{#2}{\gtrless}}}}
\newcommand{\myvec}[1]{\ensuremath{\begin{pmatrix}#1\end{pmatrix}}}
\newcommand{\mydet}[1]{\ensuremath{\begin{vmatrix}#1\end{vmatrix}}}
\numberwithin{equation}{section}
\numberwithin{figure}{section}
\numberwithin{table}{section}
\makeatletter
\@addtoreset{figure}{problem}
\makeatother
\let\StandardTheFigure\thefigure{}
\let\vec\mathbf{}
\def\putbox#1#2#3{\makebox[0in][l]{\makebox[#1][l]{}\raisebox{\baselineskip}[0in][0in]{\raisebox{#2}[0in][0in]{#3}}}}
     \def\rightbox#1{\makebox[0in][r]{#1}}
     \def\centbox#1{\makebox[0in]{#1}}
     \def\topbox#1{\raisebox{-\baselineskip}[0in][0in]{#1}}
     \def\midbox#1{\raisebox{-0.5\baselineskip}[0in][0in]{#1}}
\vspace{3cm}
\title{Assignment 6 12th Class}
\author{Gunjit Mittal (AI21BTECH11011)}
\date{\today}
\logo{\large \LaTeX}
\begin{document} 
\begin{frame}
  \titlepage{}
\end{frame}
\logo{}
\begin{frame}{Outline}
  \tableofcontents
\end{frame}
% Download all python codes from 
% \begin{lstlisting}
% https://github.com/GunjitMittal/Assignment6/tree/main/Assignment6/code
% \end{lstlisting}     
% Download all latex codes from 
% \begin{lstlisting}
% https://github.com/GunjitMittal/Assignment6/tree/main/Assignment6 
% \end{lstlisting} 
\section{Question}
\begin{frame}{Question}
x and y are independent, identically distributed (i.i.d) random variables with common p.d.f
\begin{align*}
    f_x\brak{x} = e^{-x}U\brak{x}~~~~~~~f_y\brak{y} = e^{-y}U\brak{y}
\end{align*}
Find the p.d.f of the following random variables (a) x+y, (b) x-y, (c) xy, (d) x/y, (e) min (x,y), (f) max (x,y), (g) min (x,y) / max (x,y)
\end{frame}
\section{Solution 6(a)} 
\begin{frame}{Solution 6(a)}
Define 
\begin{align}
  Z = X + Y
\end{align}
Note that both X and Y are positive random variables hence 
\begin{align}
  &f_Z\brak{z} = \int_{0}^{z} f_{XY}\brak{z-y,y}dy = \int_0^z e^{-\brak{z-y+y}}dy\\
 &~~~~~~= ze^{-z}U\brak{z}
\end{align}
\end{frame}
\section{Solution 6(b)}
\begin{frame}{Solution 6(b)}
\begin{align}
    Z = X - Y
\end{align}
Z ranges over the entire real axis for the random variables X and Y
\begin{align}
  F_Z\brak{z} = 
  \begin{cases}
    \int_0^\infty \int_0^{z+y} f_{XY}\brak{x,y}dx~dy,  & z > 0\\
    \int_{-z}^\infty \int_0^{z+y} f_{XY}\brak{x,y}dx~dy, & z < 0
  \end{cases}
\end{align}
Differrentiation gives
\begin{align}
  f_Z\brak{z} = 
  \begin{cases}
    \int_0^\infty f_{XY}\brak{z+y,y}dy,  & z > 0\\
    \int_{-z}^\infty f_{XY}\brak{z+y,y}dy, & z < 0
  \end{cases}
\end{align}
\begin{align}
  f_Z\brak{z} = 
  \begin{cases}
    \int_0^\infty e^{-\brak{z+y+y}}dy = e^{-z}\int_0^\infty e^{-2y}dy = \frac{1}{2}e^{-z}, & z > 0\\
    \int_{-z}^\infty e^{-\brak{z+y+y}}dy = e^{-z}\int_{-z}^\infty e^{-2y}dy = \frac{1}{2}e^{z},& z < 0
  \end{cases}
\end{align}
\begin{align}
  f_Z\brak{z} = \frac{1}{2}e^{-|z|},~~~~~~ -\infty \leq z \leq \infty .
\end{align}
\end{frame} 
\section{Solution 6(c)}
\begin{frame}{Solution 6(c)}
  \begin{align}
    &~~~~~~~~Z = XY\\
    &F_Z\brak{z} = P\cbrak{Z \leq z} = P\cbrak{XY \leq z }\\
    & = \int_0^\infty \int_0^{z/y} f_{XY}\brak{x,y}dx~dy\\
    &f_Z\brak{z} = \int_0^\infty \frac{1}{y}f_{XY} \brak{\frac{z}{y},y} dy = \int_0^\infty \frac{1}{y}e^{-\brak{\brak{z/y}+y}}dy
  \end{align}
\end{frame}
\section{Solution 6(d)}
\begin{frame}{Solution 6(d)}
  \begin{align}
    &Z = X/Y \\
    &F_Z\brak{z} = P\cbrak{Z \leq z} = P\cbrak{\frac{X}{Y} \leq z}\\
    &= \int_0^\infty \int_0^{yz} f_{XY}\brak{x,y}dx~dy\\
    &f_Z\brak{z} = \int_0^\infty yf_{XY} \brak{yz,y} dy = \int_0^\infty ye^{y\brak{z+1}}dy = \int_0^\infty ye^{\brak{1+z}y}dy\\
    &~~~~~~=\sbrak{y\frac{e^{-\brak{1+z}y}}{-\brak{1+z}}}_0^\infty + \brak{\frac{1}{1+z}}\int_0^\infty e^{\brak{1+z}y}dy\\
    &~~~~~~=\brak{\frac{1}{1+z}}\sbrak{\frac{e^{-\brak{1+z}y}}{-\brak{1+z}}}_0^\infty = \frac{1}{{\brak{1+z}}^2}U\brak{z}
  \end{align}
\end{frame}
\section{Solution 6(e)}
\begin{frame}{Solution 6(e)}
  \begin{align}
    &~~~~~~~~~~~Z = \text{min} \brak{X,Y}\\
    &F_Z\brak{z} = P\cbrak{\text{min}\brak{X,Y}\leq z}\\
    &~~~~~~~=1-P\cbrak{X > z, Y > z}\\
    &~~~~~~~=1-\sbrak{1-F_X\brak{z}}\sbrak{1-F_Y\brak{z}}\\
    &~~~~~~~=F_X\brak{z} + F_Y\brak{z} -F_X\brak{z}F_Y\brak{z}\\
    &f_Z\brak{z} = f_X\brak{z} + f_Y\brak{z} - F_X\brak{z}f_Y\brak{z} - f_X\brak{z}F_Y\brak{z}.
  \end{align}
  We have
  \begin{align}
    f_X\brak{z} = f_Y\brak{z} = e^{-z}U\brak{z}
  \end{align}
  so that
\end{frame}
\begin{frame}
  \begin{align}
    &F_X\brak{z} = \int_0^\infty e^{-x}dx = \brak{1-e^{-z}}U\brak{z} = F_Y\brak{z}\\
    &f_Z\brak{z} = \sbrak{e^{-z} +e^{-z}-2\brak{1-e^{-z}}e^{-z}}U\brak{z}\\
    &~~~~~~~= 2e^{-z}\sbrak{1-1+e^{-z}}U\brak{z}\\
    &~~~~~~~= 2e^{-2z}U\brak{z}
  \end{align}
\end{frame}
\section{Solution 6(f)}
\begin{frame}{Solution 6(f)}
  \begin{align}
    &~~~~~~~~~~~Z = \text{max} \brak{X,Y}\\
    &F_Z\brak{z} = P\cbrak{\text{max}\brak{X,Y}\leq z} = P\cbrak{X\leq z, Y \leq z}\\
    &~~~~~~~=P\cbrak{X \leq z}P\cbrak{Y \leq z}=F_X\brak{z}F_Y\brak{z}
  \end{align}
  \begin{align}
    &f_z\brak{z} = F_X\brak{z}f_Y\brak{z}+f_X\brak{z}F_Y\brak{z}\\
    &~~~~~~~=e^{-z}\brak{1-e^{-z}}+e^{-z}\brak{1-e^{-z}}\\
    &~~~~~~~=2e^{-z}\brak{1-e^{-z}}U\brak{z}
  \end{align}
\end{frame}
\section{Solution 6(g)}
\begin{frame}{Solution 6(g)}
  \begin{align}
    &~~~~~~~~~Z = \frac{\text{min}\brak{X,Y}}{\text{max}\brak{X,Y}}, ~~~ 0<z<1\\
    &F_Z\brak{z} = P\cbrak{\brak{\frac{\text{min}\brak{X,Y}}{\text{max}\brak{X,Y}}\leq z}\brak{\brak{X \leq Y}+\brak{X>Y}}}\\
    &= P\cbrak{\brak{\frac{\text{min}\brak{X,Y}}{\text{max}\brak{X,Y}}\leq z}\brak{X \leq Y}}+P\cbrak{\brak{\frac{\text{min}\brak{X,Y}}{\text{max}\brak{X,Y}}\leq z}\brak{X >Y}}\\
    &=P\cbrak{\frac{X}{Y}\leq z,X\leq Y}+P\cbrak{\frac{X}{Y}\leq z,X > Y}\\
    &=P\cbrak{X\leq Yz,X\leq Y}+P\cbrak{Y\leq Xz,X > Y}\\
    &=\int_0^\infty \int_0^{yz}f_{XY}\brak{x,y}dx~dy+\int_0^\infty \int_0^{xz}f_{XY}\brak{x,y}dy~dx
  \end{align}
\end{frame}
\begin{frame}
  \begin{align}
    &f_Z\brak{z} = \int_0^\infty yf_{XY}\brak{yz,y}dy +\int_0^\infty xf_{XY}\brak{x,xz}dx\\
    &~~~~~~~= \int_0^\infty yf_{XY}\brak{yz,y}dy + \int_0^\infty yf_{XY}\brak{y,yz}dy\\
    &~~~~~~~= \int_0^\infty y\brak{e^{-\brak{yz+y}}+e^{-\brak{y+yz}}}dy\\
    &~~~~~~~= 2\int_0^\infty ye^{-y\brak{1+z}}dz = 
    \begin{cases}
      \frac{2}{{\brak{1+z}}^2}, & 0\leq z \leq 1\\
      0,& \text{otherwise}
    \end{cases}
  \end{align}
\end{frame}
\end{document}     